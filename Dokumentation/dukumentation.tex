%%
%% Dokumentenart
%%
\NeedsTeXFormat{LaTeX2e}
\documentclass[
    a4paper,
    BCOR32mm,
    10pt,
    bibliography=totoc,
    twoside,
    openright,
    numbers=noenddot,
    headings=normal,
    DIV=9,
    parskip
    %,draft
]{scrbook}


%%
%% Unterstützung für deutsche Sprache, Umlaute etc.
%%
\usepackage[utf8]{inputenc}
\usepackage[T1]{fontenc}
\usepackage[ngerman]{babel}
\usepackage[babel,german=quotes]{csquotes}


%%
%% Diverse Pakete
%%
\usepackage{scrhack}
\usepackage{graphicx}
\usepackage{verbatim}
\usepackage{tabularx}
\usepackage{subfig}
\usepackage{url}
\usepackage{xcolor}
\usepackage{amssymb}
\usepackage{amsmath}
%\usepackage{amsthm}
\usepackage{setspace}
\usepackage{listings}
\usepackage{colortbl}
%\usepackage{showframe} % Seitenspiegel anzeigen
\usepackage{microtype}
%\usepackage{multicol}
%\usepackage[amsmath,thmmarks,hyperref]{ntheorem}
\usepackage{ntheorem}

\usepackage{siunitx} % SI einheiten
\sisetup{
	locale = DE ,
	per-mode = symbol
}

\usepackage{todonotes}
\usepackage[numbers]{natbib} % zusätzlicher Bibliographiestil
\usepackage{hyperref} % muss letztes Paket in der Liste sein


%%
%% hier Namen etc. einsetzen
%%
\newcommand{\fullname}{Florian, Samuel, Jonas}
\newcommand{\email}{oberle@mail.hs-ulm.de}
\newcommand{\titel}{Projekt-CSE-openFOAM}
\newcommand{\subtitel}{}
\newcommand{\jahr}{2016}
\newcommand{\matnr}{}
%\newcommand{\betreuer}{}
%\newcommand{\fakultaet}{Ingenieurwissenschaften, Informatik und Psychologie}
\newcommand{\fakultaet}{Mathematik und Wirtschaftswissenschaften}
%\newcommand{\fakultaet}{Naturwissenschaften}
%\newcommand{\fakultaet}{Medizin}
%\newcommand{\institut}{Institut für Irgendetwas}
%\newcommand{\arbeit}{Diplomarbeit}
\newcommand{\arbeit}{Projektarbeit}
%\newcommand{\arbeit}{Masterarbeit}


%%
%% Setzt Autor und Titel in den Metadaten des erzeugten Dokumentes
%%
\pdfinfo{
    /Author (\fullname)
    /Title (\titel)
    /Producer (pdfeTex 3.14159-1.30.6-2.2)
    /Keywords ()
}
\hypersetup{
    pdftitle=\titel,
    pdfauthor=\fullname,
    pdfsubject={\arbeit},
    pdfproducer={pdfeTex 3.14159-1.30.6-2.2},
    colorlinks=false,
    pdfborder=0 0 0
}


%%
%% Tiefe, bis zu der Überschriften in das Inhaltsverzeichnis kommen
%%
\setcounter{tocdepth}{3}


%%
%% Verhindert überhängende Absatzteile
%%
\clubpenalty10000
\widowpenalty10000
\displaywidowpenalty=10000


%%
%% Einstellungen für Codelistings
%%
\lstset{
    language=C++,
    showstringspaces=false,
    frame=single,
    numbers=left,
    basicstyle=\ttfamily,
    numberstyle=\tiny
}


%%
%% Formatierung des Literaturverzeichnisses
%%
%\bibliographystyle{plaindin} % Nummern und alphabetisch sortiert
%\bibliographystyle{plainnat} % Nummern und alphabetisch sortiert, URL, DOI etc. in der Angabe
\bibliographystyle{alphadin} % Buchstaben und sortiert
%\bibliographystyle{abbrvdin} % Nummern und abgekürzte Namen
%\bibliographystyle{unsrtdin} % Nummern und unsortiert


%%
%% Eigene Makros
%%

%-------------------------------------------------------------------------------
%                Deklaration der Bemerkungsumgebung
%					benötigt das theorem paket
%-------------------------------------------------------------------------------
\theoremstyle{marginbreak}
\theoremseparator{:}
%\theorembodyfont{\notefont}
%\theoremheaderfont{\notefonthead}
\newtheorem{defi}{Definition}[subsection]


%%
%% Eigene Farben
%%
\definecolor{Gray}{rgb}{0.80784, 0.86667, 0.90196} %dunkelblau
\definecolor{Lightgray}{rgb}{0.9176, 0.95, 0.95686} %hellblau
\definecolor{Akzent}{rgb}{0.6627, 0.63529, 0.55294} %akzentfarbe


%%
%% Liniendicke in Tabellen etc.
%%
\setlength{\arrayrulewidth}{0.1pt}


%%
%% Schriftarten
%%
\renewcommand{\sfdefault}{phv}
\renewcommand{\rmdefault}{phv}
\renewcommand{\ttdefault}{pcr}
\KOMAoptions{DIV=last}


%%
%% Seitenlayout
%%
\pagestyle{headings}


%%
%% Trennungsregeln
%%
\hyphenation{Sil-ben-trenn-ung}


%%
%% Schönere Bullets bei Aufzählungen
%%
\renewcommand{\labelitemi}{$\bullet$}
\renewcommand{\labelitemii}{$\circ$}
\renewcommand{\labelitemiii}{$\cdot$}


%%
%% Beginn des eigentlichen Dokumentes
%%
\begin{document}


%%
%% Vorspann
%%
\frontmatter


%%
%% Titelseite
%%
\thispagestyle{empty}
\begin{addmargin*}[4mm]{-32mm}
    % Logo und Wortmarke
    \begin{center}
	    %\includegraphics[height=8cm]{images/hslogo}
    \end{center}
    
    \vspace*{4.1em}

    % Briefkopf
    \footnotesize
    \textbf{Universität Ulm} \textbar ~89075 Ulm \textbar ~Germany
    \hfill
    %\parbox[t]{42mm}{\bfseries \raggedright Fakultät für \fakultaet\\\mdseries\institut}
    \vspace*{2cm}

    % Titel
    \parbox{140mm}{\bfseries \raggedright \huge \titel\\[1ex]
			    	\Large \subtitel}

    % Untertitel
    {\arbeit{}}
    \vspace*{4em}

    % Prüfer etc.
    \textbf{Vorgelegt von:}\\\fullname\\\email\\[2em]
    %\textbf{Betreuer:}\\\betreuer\\[1.5em]
    \jahr
\end{addmargin*}


%%
%% Impressum
%%
\clearpage
\thispagestyle{empty}
{
    % vollständiger Titel
    \small \flushleft \enquote{\titel}\\
    Fassung vom \today
    \vfill


    % Urheberrechtshinweis
    \copyright{} \jahr{} \fullname{}\\[0.5em]
    % Falls keine Lizenz gewünscht wird bitte den folgenden Text entfernen.
    % Die Lizenz erlaubt es zu nichtkommerziellen Zwecken die Arbeit zu
    % vervielfältigen und Kopien zu machen. Dabei muss aber immer der Autor
    % angegeben werden. Eine kommerzielle Verwertung ist für den Autor
    % weiter möglich.
    Dieses Werk ist unter der Creative Commons Attribution-NonCommercial-ShareAlike 3.0 Germany License lizensiert: \url{http://creativecommons.org/licenses/by-nc-sa/3.0/de/}\\
    Satz: PDF-\LaTeXe{}\\
}

%%
%% Inhaltsverzeichnis
%%
\setstretch{1.4}
\tableofcontents
%\listoftodos %hiermit wird ähnlich dem Inhaltsverzeichnis eine Liste aller offenen ToDo Einträge inkl. Link und Seitenzahl erstellt 

%%
%% Hauptteil
%%
\mainmatter

\chapter{Beginn der Dokumentation}

\section{Bemerkungen}

Beim Lesen des Tutorials "`Cavity"' sind folgende Dinge aufgetaucht, die nicht jeder extra nachschlagen muss:\\

\begin{defi}[kenemtic pressure]
	Wird an Stelle des Drucks p verwendet. Damit lässt sich die Navier Stokes Gleichung ohne die Dichte ausdrücken.
	\[P = \frac{p}{\varrho}\]
	Hat die Einheit \si{\square\meter\per\square\second}
\end{defi}



% hier weitere Kapitel einbinden


%%
%% Anhänge
%%
%\appendix
%\input{chapters/sources}


%%
%% Nachspann
%%
\backmatter

\listoffigures

%%
%% Literaturverzeichnis
%%
\bibliography{mybibliography}


%%
%% Versicherung
%%
\cleardoublepage
\thispagestyle{empty}

Name: \fullname \hfill Matrikelnummer: \matnr \vspace{2cm}

\minisec{Erklärung}

Ich erkläre, dass ich die Arbeit selbständig verfasst und keine anderen als die angegebenen Quellen und Hilfsmittel verwendet habe.\vspace{2cm}

Ulm, den \dotfill

\hfill {\footnotesize \fullname}
\end{document}
